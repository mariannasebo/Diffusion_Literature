% Options for packages loaded elsewhere
\PassOptionsToPackage{unicode}{hyperref}
\PassOptionsToPackage{hyphens}{url}
\PassOptionsToPackage{dvipsnames,svgnames,x11names}{xcolor}
%
\documentclass[
  letterpaper,
  DIV=11,
  numbers=noendperiod]{scrartcl}

\usepackage{amsmath,amssymb}
\usepackage{lmodern}
\usepackage{iftex}
\ifPDFTeX
  \usepackage[T1]{fontenc}
  \usepackage[utf8]{inputenc}
  \usepackage{textcomp} % provide euro and other symbols
\else % if luatex or xetex
  \usepackage{unicode-math}
  \defaultfontfeatures{Scale=MatchLowercase}
  \defaultfontfeatures[\rmfamily]{Ligatures=TeX,Scale=1}
\fi
% Use upquote if available, for straight quotes in verbatim environments
\IfFileExists{upquote.sty}{\usepackage{upquote}}{}
\IfFileExists{microtype.sty}{% use microtype if available
  \usepackage[]{microtype}
  \UseMicrotypeSet[protrusion]{basicmath} % disable protrusion for tt fonts
}{}
\makeatletter
\@ifundefined{KOMAClassName}{% if non-KOMA class
  \IfFileExists{parskip.sty}{%
    \usepackage{parskip}
  }{% else
    \setlength{\parindent}{0pt}
    \setlength{\parskip}{6pt plus 2pt minus 1pt}}
}{% if KOMA class
  \KOMAoptions{parskip=half}}
\makeatother
\usepackage{xcolor}
\usepackage[top=30mm,left=30mm]{geometry}
\setlength{\emergencystretch}{3em} % prevent overfull lines
\setcounter{secnumdepth}{-\maxdimen} % remove section numbering
% Make \paragraph and \subparagraph free-standing
\ifx\paragraph\undefined\else
  \let\oldparagraph\paragraph
  \renewcommand{\paragraph}[1]{\oldparagraph{#1}\mbox{}}
\fi
\ifx\subparagraph\undefined\else
  \let\oldsubparagraph\subparagraph
  \renewcommand{\subparagraph}[1]{\oldsubparagraph{#1}\mbox{}}
\fi


\providecommand{\tightlist}{%
  \setlength{\itemsep}{0pt}\setlength{\parskip}{0pt}}\usepackage{longtable,booktabs,array}
\usepackage{calc} % for calculating minipage widths
% Correct order of tables after \paragraph or \subparagraph
\usepackage{etoolbox}
\makeatletter
\patchcmd\longtable{\par}{\if@noskipsec\mbox{}\fi\par}{}{}
\makeatother
% Allow footnotes in longtable head/foot
\IfFileExists{footnotehyper.sty}{\usepackage{footnotehyper}}{\usepackage{footnote}}
\makesavenoteenv{longtable}
\usepackage{graphicx}
\makeatletter
\def\maxwidth{\ifdim\Gin@nat@width>\linewidth\linewidth\else\Gin@nat@width\fi}
\def\maxheight{\ifdim\Gin@nat@height>\textheight\textheight\else\Gin@nat@height\fi}
\makeatother
% Scale images if necessary, so that they will not overflow the page
% margins by default, and it is still possible to overwrite the defaults
% using explicit options in \includegraphics[width, height, ...]{}
\setkeys{Gin}{width=\maxwidth,height=\maxheight,keepaspectratio}
% Set default figure placement to htbp
\makeatletter
\def\fps@figure{htbp}
\makeatother
\newlength{\cslhangindent}
\setlength{\cslhangindent}{1.5em}
\newlength{\csllabelwidth}
\setlength{\csllabelwidth}{3em}
\newlength{\cslentryspacingunit} % times entry-spacing
\setlength{\cslentryspacingunit}{\parskip}
\newenvironment{CSLReferences}[2] % #1 hanging-ident, #2 entry spacing
 {% don't indent paragraphs
  \setlength{\parindent}{0pt}
  % turn on hanging indent if param 1 is 1
  \ifodd #1
  \let\oldpar\par
  \def\par{\hangindent=\cslhangindent\oldpar}
  \fi
  % set entry spacing
  \setlength{\parskip}{#2\cslentryspacingunit}
 }%
 {}
\usepackage{calc}
\newcommand{\CSLBlock}[1]{#1\hfill\break}
\newcommand{\CSLLeftMargin}[1]{\parbox[t]{\csllabelwidth}{#1}}
\newcommand{\CSLRightInline}[1]{\parbox[t]{\linewidth - \csllabelwidth}{#1}\break}
\newcommand{\CSLIndent}[1]{\hspace{\cslhangindent}#1}

\KOMAoption{captions}{tableheading}
\makeatletter
\makeatother
\makeatletter
\makeatother
\makeatletter
\@ifpackageloaded{caption}{}{\usepackage{caption}}
\AtBeginDocument{%
\ifdefined\contentsname
  \renewcommand*\contentsname{Table of contents}
\else
  \newcommand\contentsname{Table of contents}
\fi
\ifdefined\listfigurename
  \renewcommand*\listfigurename{List of Figures}
\else
  \newcommand\listfigurename{List of Figures}
\fi
\ifdefined\listtablename
  \renewcommand*\listtablename{List of Tables}
\else
  \newcommand\listtablename{List of Tables}
\fi
\ifdefined\figurename
  \renewcommand*\figurename{Figure}
\else
  \newcommand\figurename{Figure}
\fi
\ifdefined\tablename
  \renewcommand*\tablename{Table}
\else
  \newcommand\tablename{Table}
\fi
}
\@ifpackageloaded{float}{}{\usepackage{float}}
\floatstyle{ruled}
\@ifundefined{c@chapter}{\newfloat{codelisting}{h}{lop}}{\newfloat{codelisting}{h}{lop}[chapter]}
\floatname{codelisting}{Listing}
\newcommand*\listoflistings{\listof{codelisting}{List of Listings}}
\makeatother
\makeatletter
\@ifpackageloaded{caption}{}{\usepackage{caption}}
\@ifpackageloaded{subcaption}{}{\usepackage{subcaption}}
\makeatother
\makeatletter
\@ifpackageloaded{tcolorbox}{}{\usepackage[many]{tcolorbox}}
\makeatother
\makeatletter
\@ifundefined{shadecolor}{\definecolor{shadecolor}{rgb}{.97, .97, .97}}
\makeatother
\makeatletter
\makeatother
\ifLuaTeX
  \usepackage{selnolig}  % disable illegal ligatures
\fi
\IfFileExists{bookmark.sty}{\usepackage{bookmark}}{\usepackage{hyperref}}
\IfFileExists{xurl.sty}{\usepackage{xurl}}{} % add URL line breaks if available
\urlstyle{same} % disable monospaced font for URLs
\hypersetup{
  pdftitle={Literature},
  pdfauthor={Marianna Sebő},
  colorlinks=true,
  linkcolor={blue},
  filecolor={Maroon},
  citecolor={Blue},
  urlcolor={Blue},
  pdfcreator={LaTeX via pandoc}}

\title{Literature}
\author{Marianna Sebő}
\date{3/14/23}

\begin{document}
\maketitle
\ifdefined\Shaded\renewenvironment{Shaded}{\begin{tcolorbox}[borderline west={3pt}{0pt}{shadecolor}, boxrule=0pt, breakable, frame hidden, sharp corners, enhanced, interior hidden]}{\end{tcolorbox}}\fi

\renewcommand*\contentsname{Table of contents}
{
\hypersetup{linkcolor=}
\setcounter{tocdepth}{3}
\tableofcontents
}
The aim of this article is to present the state of the art on Policy
Diffusion and Innovation in Public Policy.

Our topic lies in the intersection of policy diffusion, innovation and
public policy. Hence, we draw from the policy diffusion theories from
Political Science, the literature on the policy reforms - especially on
public goods and services - from Economics. Last, insights on
governance, coordination and costs will be also drawn from Public
Administration. The combination of these fields allows as to explain
this research from a multi-model point of view.

Policy innovation happens when a government adopts a new policy (Shipan
and Volden 2008). The mechanisms of the policy adoption can be diverse.
The policy innovation can be endogeneous, coming from within the
government, the residents or local stakeholders. Also, policy innovation
can be a result of policy diffusion, e.g.~the result of the spread of
adoptions and innovations.

(Jackson and Yariv 2011) In an overview on social networks and
diffusion. Structures can influence economic behavior, diffusion of
behavior and policies. Epidemiological models can be useful to model
economic phenomena, as members interact. Relevant to our research, units
or agent might be interested about the proportion of units adopting a
given action. For their decision, when a given adoption threshold has
been reached, their incentive might increase and they might want to take
an action - same or different - from that of the neighbors. Agents that
don't adopt early policies can free-ride on the information on the
results of the actions of the neighbors. In empirical literature
diffusion has been analyzed through the lenses of social networks on
several fields. These include: marketing, labor economics, political
economy, etc.

The components of the network and the connections between them are
important in applications such as contagion, learning and diffusion.
Different subgraphs of the network are the components. Hence, if we
consider Catalonia as the whole network, the several components are
municipalities that form connections between them. These components are
network partitions forming path-connected groups of nodes. A network
that consists of only one component is a connected network. A network
that has several components but doesn't have a cycle (e.g.~several
trees) is a forest. A special case of a forest is a star network where
one node acts as a center, e.g.~every link of the network involves that
node. The set of nodes that the node \(i\) Is linked to is called the
neighborhood.

The \emph{degree} of a node is the number of links involving that node.

Early literature about diffusion: Hybrid con adoption Ryan and Gross
1943, Griliches 1957, Drug adoption Coleman Katz, Menzel 1966

Importance of social connections

``opinion leaders'' in the study of voting

Whereas, as typically in economics, there is a growing literature with
studies on correlations, casual inference might not be possible without
a specific setting. Therefor several authors use experiments as in

Various field experiments, such as those by Duflo and Saez (2003),
Karlan, Mobius, Rosenblat, and Szeidl (2009), Dupas (2010), Beaman and
Magruder (2010), and Feigenberg, Field, and Pande (2010),

Another way to reach casual inference is using sctructural modeling.
Banerjee, Chandrasekhar, Duflo, and Jackson (2010)

Newman, Barabasi \& Watts (2006),

The Handbook of Social Economics (forthcoming)

There are some popular texts such as Watts (2003) and Barabasi (2004),
as well as a history of thought of the sociology literature by Freeman
(2006) .

Goyal (2008)

Jackson (2008) synthesizes the analyses of networks from sociology,
economics, statistical physics, mathematics, and computer science.

Greenhouse gas emissions trading (ET) system has been an emerging policy
to govern global issues. In this setting, multiple authorities are part
of a governance system, whereas they scale and interconnectedness might
differ, leading to a polycentric setting Ostrom 2010 a 2010b

Social network analysis arose in Sociology (e.g., Boissevain \&
Mitchell, 1973; Coleman, 1958; Scott, 1991; Wasserman \& Faust, 1997;
Wellman, 1983) but has recently emerged as a crucial methodology in
political science as well (e.g., Bach \& Newman, 2010; Cao, 2009, 2010;
HafnerBurton, Kahler, \& Montgomery, 2009; Hafner-Burton \& Montgomery,
2006; Ward, Stovel, \& Sacks, 2011).

The setting

The set \(N = {1, ..., n}\) is the set of nodes that form part of the
network. These nodes in our contexts are municipalities of Catalonia.
Two nodes are either connected or not, - they cooperate or they don't.
IMC is a reciprocal relationship, all the participants that form part of
it have to agree to it. Such relationship can be modelled as un
undirected network. Then, we have a graph \((N, g)\) that consists of a
set of nodes \(N = {1, ..., n}\) and a \$n x n\$ matrix \(g\) where
\(g_{ij}\) stands for the relationship between the nodes. Such
relationship can be weighted or unweighted (Jackson 2008). To represent
the map of IMC in Catalonia as a network is appealing for several
reasons. IMC is a relationship that migh be advantageous or
disadvantageous. Such relationship can be quantified by common
statistical parameter, e.g.: correlations, covariances, regression
coefficients, partial correlations etc. Such connections can be seen as
lin ks and modelled as paths through the network.

Partial correlation networks can provide valuable hypothesis generating
structures, which may reflect potential causal effects to be further
examined in terms of conditional independence (Pearl, 2000). When
continuous data are multivariate normally distributed, analysing the
partial correlations using the Gaussian graphical model (GGM; Costantini
et al., 2015; Lauritzen, 1996) is appropriate. If the continuous data
are not normally distributed then a transformation (e.g.~nonparanormal
transformation, Liu, Lafferty, \& Wasserman, 2009) can be applied prior
to applying the GGM. The GGM can also be used for ordinal data, wherein
the network is based on the polychoric correlations instead of partial
correlations (Epskamp, 2018). If all the research variables are binary,
the Ising Model can be used (van Borkulo et al., 2014). When the data
comprise a mixture of categorical and continuous variables, the Mixed
Graphical Model can be used to estimate the PMRF (Haslbeck \& Waldorp,
2016). Thus, networks can be estimated from various types of data in a
flexible manner. Inter-municipal cooperation as a delivery mode is by
definition a creates network(s) of local public service delivery in the
context of network of municipalities. It has been argued in the
literature WARNER, that IMC is complex process of cooperation where
members exchange information and learn from each other. From the point
of view of the literature of policy diffusion, the question arises how
such interdependent structures shapes new policies and regulations. The
already existing structures can be affected by processes that include
learning, cooperation, competition etc. In case of countryes being
affected by policy diffusion, policy diffusion is mainly characterized
by the fact that a policies in a given country are systematically
conditions by prior policies of another one Simmons et al.~(2006, 787).

(Boehmke et al. 2020) Walkers 1969 pioneering study.

(Gray 1973) Starting from Gray (1973), scholars have been interested in
policy from a multidimensional point of view. The author asked questions
such as how new ideas diffuse, why some are more innovative than others
and whether there are certain patterns of innovation. At the time of
this research, the frequency of adoption of policies showed an ``S''
shape, which is similar to an individual's learning curve. Adopters and
non-adopters interact. Following this work, the rate of spread of
adoptions can be expressed by

\[
\DeltaA_t = f(A_t)
\]

In this notation, \(A_t\) is the cumulative proportion of adopters of
year \(t\) and \(\Delta A_t = A_{t+1} - A_t\) (see Gray 1973; Berry and
Berry 1990) f policy convergence (Bennett 1991) s (Holzinger et
al.~2008). y. Spatial econometric modeling has become the method of
choice in the literature (Franzese and Hays 2007; Ward and Gleditsch
2008).

Statistical aspects of analyzing netwokr data: Descriptive Statistics of
the Networks: Books: • Kolaczyk, E. \& Csardi, G. (2014): Statistical
Analysis of Network Data with R, Springer. • Newman, N. (2018):
Networks, Oxford University Press. • Salter-Townshend, M., White, A.,
Gollini, I., \& Murphy, T. B. (2012): Review of statistical network
analysis: models, algorithms, and software, Statistical Analysis and
Data Mining, 5(4), 243-264.

(Shipan and Volden 2008) Subnational governments of federal systems, the
important role of municipalities, and decentralization serve as an
interesting opportunity to experiment. Experimenting with policies can
be advantageous for varios reasons. Given the same national jurisdiction
and similar context, the possibility to experiment with policies can be
observed by other municipalities. The leaders that adopt first such
policies and the laggerds - who adopt them later - are not independent
from each other. Neighborhood affects arise, e.g.: learning,
competition, imitation or coercion. Whether the adoption of policy is a
result of one of these mechanisms can tell a lot about the suitability
and the expected effects of the policy. Policies, that result from
learning, might be more effective. When learning, a municipality
observes the outcome of the policy adopted by the leader. Based on the
observed outcomes decides whether to introduce the policy or not. On the
other hand, imitation doesn't focus on the result of the policy, if
focuses on the municipality that adopted the policy. This is normally
the case of small municipalities that look at what a bigger municipality
of the neighborhood is doing. Importantly, imitation is not expected to
produce the desirable outcome if the municipality that imitates and that
is being imitated are different and not directly comparable. Coercion is
a mechanims induced by a higher-tier government which -if not all the
municipalities are equal- is not expected to be optimal. Last, scholars
distinguish competition, which can be either advantageous or
disadvantageous depending the spillover effects in place. The authors of
this paper investigate the antismoking laws across the US based on 675
largest cities over the period of 1975 and 2000. As expected, the
leaders tend to be bigger cities, for the smaller ones learning is
difficult, they have a hihger probability to engage in competition or
imitation. They have also less influence on higher-tier decision -
coercion can be an issue for them.

\hypertarget{refs}{}
\begin{CSLReferences}{1}{0}
\leavevmode\vadjust pre{\hypertarget{ref-RN2592}{}}%
Boehmke, Frederick J., Mark Brockway, Bruce A. Desmarais, Jeffrey J.
Harden, Scott LaCombe, Fridolin Linder, and Hanna Wallach. 2020.
{``SPID:~A New Database for Inferring Public Policy Innovativeness and
Diffusion Networks.''} Journal Article. \emph{Policy Studies Journal} 48
(2): 517--45. https://doi.org/\url{https://doi.org/10.1111/psj.12357}.

\leavevmode\vadjust pre{\hypertarget{ref-RN2591}{}}%
Gray, Virginia. 1973. {``Innovation in the States: A Diffusion Study.''}
Journal Article. \emph{The American Political Science Review} 67 (4):
1174--85. \url{https://doi.org/10.2307/1956539}.

\leavevmode\vadjust pre{\hypertarget{ref-RN2590}{}}%
Jackson, Matthew O. 2008. {``Representing and Measuring Networks.''}
Book Section. In \emph{Social and Economic Networks}, 20--53. Princeton
University Press. \url{https://doi.org/10.2307/j.ctvcm4gh1.5}.

\leavevmode\vadjust pre{\hypertarget{ref-RN2587}{}}%
Jackson, Matthew O., and Leeat Yariv. 2011. {``Chapter 14 - Diffusion,
Strategic Interaction, and Social Structure.''} Book Section. In
\emph{Handbook of Social Economics}, edited by Jess Benhabib, Alberto
Bisin, and Matthew O. Jackson, 1:645--78. North-Holland.
https://doi.org/\url{https://doi.org/10.1016/B978-0-444-53187-2.00014-0}.

\leavevmode\vadjust pre{\hypertarget{ref-RN2595}{}}%
Shipan, Charles R., and Craig Volden. 2008. {``The Mechanisms of Policy
Diffusion.''} Journal Article. \emph{American Journal of Political
Science} 52 (4): 840--57.
https://doi.org/\url{https://doi.org/10.1111/j.1540-5907.2008.00346.x}.

\end{CSLReferences}



\end{document}
